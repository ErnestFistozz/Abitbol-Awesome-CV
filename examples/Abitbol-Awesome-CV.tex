%!TEX TS-program = xelatex
%!TEX encoding = UTF-8 Unicode
% Abitbol-Awesome-CV LaTeX Template pour CV/Resume
%
% This original template has been downloaded from:
% https://github.com/posquit0/Awesome-CV
%
% Author:
% Claud D. Park <posquit0.bj@gmail.com>
% http://www.posquit0.com
%
% Ce model a été téléchargé depuis:
% https://github.com/ordinatous/Abitbol-Awesome-CV
%
% Auteur:
% ordinatous <contact@ordinatous.com>
% https://ordinatous.com
%
% Template license:
% CC BY-SA 4.0 (https://creativecommons.org/licenses/by-sa/4.0/)
%
%-------------------------------------------------------------------------------
% CONFIGURATIONS du document
%-------------------------------------------------------------------------------
% FORMAT du papier
% Format A4 par défaut, utiliser 'letterpaper' pour le format US letter:
\documentclass[11pt, a4paper]{awesome-cv}
% LANGUE
% Francisation du document , format de date, syntaxe et correcteur orthographique:
\usepackage[french]{babel}
% MARGES
% Configurer les marges ici:
\geometry{left=1.4cm, top=.8cm, right=1.4cm, bottom=1.8cm, footskip=.5cm}
% FONTS
% Indiquer l'emplacement des fonts ici:
\fontdir[..fonts/]
% COULEUR
% Couleur pour les mise en avant:
% Awesome Couleurs: awesome-emerald, awesome-skyblue, awesome-red, awesome-pink, awesome-orange
%                 awesome-nephritis, awesome-concrete, awesome-darknight
\colorlet{awesome}{awesome-red}
% Décommenter pour choisir d'autres couleurs:
% \definecolor{awesome}{HTML}{CA63A8}

% Coleurs pour le texte:
% Décommenter pour en choisir une:
% \definecolor{darktext}{HTML}{414141}
% \definecolor{text}{HTML}{333333}
% \definecolor{graytext}{HTML}{5D5D5D}
% \definecolor{lighttext}{HTML}{999999}
% ACTIVER Highlights
% Basculer la valeur à false pour suprimmer les mise en avant:
\setbool{acvSectionColorHighlight}{true}

% If you would like to change the social information separator from a pipe (|) to something else
\renewcommand{\acvHeaderSocialSep}{\quad\textbar\quad}


%-------------------------------------------------------------------------------
%	INFORMATIONS PERSONNELLE
%	Commenter les entrées suivantes si elles ne sont pas nécessaires:
%-------------------------------------------------------------------------------
% Available options: circle|rectangle,edge/noedge,left/right
\photo{../examples/profil_02}
\name{George}{Abitbol}
\position{Cosmonaute le plus Classe du Monde{\enskip\cdotp\enskip}Homme le plus Classe du Monde}
\address{Kennedy Space Center, Cap Canaveral, USA}

\mobile{(+82) 10-9030-1843}
\email{george@abitbol.org}
\homepage{www.abitbol.org}
% RESEAUX SOCIAUX
\github{abitbol}
\linkedin{abitbol}
% \gitlab{gitlab-id}
% \stackoverflow{SO-id}{SO-name}
% \twitter{@twit}
% \skype{skype-id}
% \reddit{reddit-id}
% \medium{madium-id}
% \googlescholar{googlescholar-id}{name-to-display}
%% \firstname and \lastname will be used
% \googlescholar{googlescholar-id}{}
% \extrainfo{extra informations}
% CITATION
\quote{``Monde de merde !"}


%-------------------------------------------------------------------------------
\begin{document}

% Placer l'entête au dessus des informations personnelle:
% options pour changer l'allignement (C: center, L: left, R: right)
\makecvheader[L]

% Print the footer with 3 arguments(<left>, <center>, <right>)
% Leave any of these blank if they are not needed
\makecvfooter
  {Le \today}
  {George Abitbol~~~·~~~Curriculum Vitae}
  {\thepage}


%-------------------------------------------------------------------------------
%	CONTENU du CV
%	Les sections sont placées dans les répertoires cv
%   modifier ces fichiers pour indiquer diplmoes, compétences, expériences et occupations:
%-------------------------------------------------------------------------------
%-------------------------------------------------------------------------------
%	SECTION TITLE
%-------------------------------------------------------------------------------
\cvsection{Titres, Diplomes ou formation }


%-------------------------------------------------------------------------------
%	CONTENT
%-------------------------------------------------------------------------------
\begin{cventries}

%---------------------------------------------------------
  \cventry
    {B.S. in Computer Science and Engineering} % Degree
    {POSTECH(Pohang University of Science and Technology)} % Institution
    {Pohang, S.Korea} % Location
    {Mar. 2010 - Aug. 2017} % Date(s)
    {
      \begin{cvitems} % Description(s) bullet points
        \item {Got a Chun Shin-Il Scholarship which is given to promising students in CSE Dept.}
      \end{cvitems}
    }

%---------------------------------------------------------
\end{cventries}

%-------------------------------------------------------------------------------
%	SECTION TITLE
%-------------------------------------------------------------------------------
\cvsection{Compétences}


%-------------------------------------------------------------------------------
%	CONTENT
%-------------------------------------------------------------------------------
\begin{cvskills}

%---------------------------------------------------------
  \cvskill
    {DevOps} % Category
    {AWS, Docker, Kubernetes, Rancher, Vagrant, Packer, Terraform, Jenkins, CircleCI} % Skills

%---------------------------------------------------------
  \cvskill
    {Back-end} % Category
    {Koa, Express, Django, REST API} % Skills

%---------------------------------------------------------
  \cvskill
    {Front-end} % Category
    {Hugo, Redux, React, HTML5, LESS, SASS} % Skills

%---------------------------------------------------------
  \cvskill
    {Programming} % Category
    {Node.js, Python, JAVA, OCaml, LaTeX} % Skills

%---------------------------------------------------------
  \cvskill
    {Languages} % Category
    {Korean, English, Japanese} % Skills

%---------------------------------------------------------
\end{cvskills}

%-------------------------------------------------------------------------------
%	SECTION TITLE
%-------------------------------------------------------------------------------
\cvsection{Expériences}


%-------------------------------------------------------------------------------
%	CONTENT
%-------------------------------------------------------------------------------
\begin{cventries}

%---------------------------------------------------------
  \cventry
    {Cosmonaute le plus Classe du Monde} % Job title
    {Kennedy Space Center.} % Organization
    { Cap Canaveral, USA} % Location
    {Jun. 2017 - May. 2018} % Date(s)
    {
      \begin{cvitems} % Description(s) of tasks/responsibilities
        \item {Provisioned an easily managable hybrid infrastructure(Amazon AWS + On-premise) utilizing IaC(Infrastructure as Code) tools like Ansible, Packer and Terraform.}
        \item {Built fully automated CI/CD pipelines on CircleCI for containerized applications using Docker, AWS ECR and Rancher.}
        \item {Designed an overall service architecture and pipelines of the Machine Learning based Fashion Tagging API SaaS product with the micro-services architecture.}
        \item {Implemented several API microservices in Node.js Koa and in the serverless AWS Lambda functions.}
        \item {Deployed a centralized logging environment(ELK, Filebeat, CloudWatch, S3) which gather log data from docker containers and AWS resources.}
        \item {Deployed a centralized monitoring environment(Grafana, InfluxDB, CollectD) which gather system metrics as well as docker run-time metrics.}
      \end{cvitems}
    }

%---------------------------------------------------------
  \cventry
    {Cosmonaute le plus Classe du Monde} % Job title
    {Kennedy Space Center.} % Organization
    { Cap Canaveral, USA} % Location
    {Jan. 2016 - Jun. 2017} % Date(s)
    {
      \begin{cvitems} % Description(s) of tasks/responsibilities
        \item {Implemented RESTful API server for car rental booking application(CARPLAT in Google Play).}
        \item {Built and deployed overall service infrastructure utilizing Docker container, CircleCI, and several AWS stack(Including EC2, ECS, Route 53, S3, CloudFront, RDS, ElastiCache, IAM), focusing on high-availability, fault tolerance, and auto-scaling.}
        \item {Developed an easy-to-use Payment module which connects to major PG(Payment Gateway) companies in Korea.}
      \end{cvitems}
    }

%---------------------------------------------------------
  \cventry
    {Cosmonaute le plus Classe du Monde} % Job title
    {Kennedy Space Center} % Organization
    { Cap Canaveral, USA} % Location
    {Mar. 2016 - Exp. Jun. 2017} % Date(s)
    {
      \begin{cvitems} % Description(s) of tasks/responsibilities
        \item {Researched classification algorithms(SVM, CNN) to improve accuracy of human exercise recognition with wearable device.}
        \item {Developed two TIZEN applications to collect sample data set and to recognize user exercise on SAMSUNG Gear S.}
      \end{cvitems}
    }

%---------------------------------------------------------
  \cventry
    {Homme le plus Classe du Monde} % Job title
    {Kennedy Space Center, MND} % Organization
    { Cap Canaveral, USA} % Location
    {Aug. 2014 - Apr. 2016} % Date(s)
    {
      \begin{cvitems} % Description(s) of tasks/responsibilities
        \item {Lead engineer on agent-less backtracking system that can discover client device's fingerprint(including public and private IP) independently of the Proxy, VPN and NAT.}
        \item {Implemented a distributed web stress test tool with high anonymity.}
        \item {Implemented a military cooperation system which is web based real time messenger in Scala on Lift.}
      \end{cvitems}
    }



%---------------------------------------------------------
 
%---------------------------------------------------------
 
%---------------------------------------------------------
\end{cventries}

%-------------------------------------------------------------------------------
%	SECTION TITLE
%-------------------------------------------------------------------------------
\cvsection{Hobbies}


%-------------------------------------------------------------------------------
%	CONTENT
%-------------------------------------------------------------------------------
\begin{cvskills}

%---------------------------------------------------------
  \cvskill
  	{Open-Source}
  	{Membre de l' \textbf{APRIL}, étude des dockers, Monitoring réseau (Suricata - SELKS) Elastisearch, InfluxDB Linuxien depuis 2008 }
%---------------------------------------------------------
  \cvskill
    {Montagne}
    {Escalade, Randonnée, ski-alpinisme, Botanique, Géomorphologie}
%---------------------------------------------------------
 
%---------------------------------------------------------
\end{cvskills}

%\input{cv/extracurricular.tex}
%\input{cv/honors.tex}
%\input{cv/presentation.tex}
%\input{cv/writing.tex}
%\input{cv/committees.tex}


%-------------------------------------------------------------------------------
\end{document}
